\documentclass[article,oneside,a4paper]{memoir}

\usepackage[utf8]{inputenc}
\usepackage[sc]{mathpazo}
\usepackage{xspace}
\usepackage{listings}
\usepackage{url}

\newcommand{\pulerau}{\texttt{pulerau}\xspace}

\begin{document}

\chapter{Guide til webgruppens arbejde}

\section{GitHub}

Udvikling af tutorgruppens hjemmeside og mailsystem fungerer via GitHub. Webfar
håndterer integration af kode i produktionen og stiller udviklingssites på
\pulerau til rådighed for de i webgruppen der ønsker det.

\section{Generel vejledning til git}

\paragraph{Identitet}
Det første, man gør, når man begynder at bruge Git, er at fortælle Git sit navn
og sin emailadresse. For eksempel:
\begin{lstlisting}
git config --global user.name 'Mathias Rav'
git config --global user.email 'rav@cs.au.dk'
\end{lstlisting}
Dette skal gøres på alle computere, hvorfra man udvikler, og det er vigtigt at
man altid bruger den samme identitet (dvs. navn og email).

\paragraph{Brug af Git}
Git er en del af webgruppens arbejde; \url{http://gitref.org/} er en glimrende
guide til at komme igang med brug af Git. Læs siden igennem fra start til slut.

\section{Vision}

Målet er en hjemmeside og et mailsystem med en overskuelig kodebase.
Det gamle mailsystem, som var i brug fra 2007 til 2012, var 2729 linjers
write-once read-never PHP.

Lamson er valgt til det nuværende mailsystem, fordi det skjuler alle de grumme
detaljer om SMTP.  Det betyder, at det er tilstrækkeligt med et par hundrede
linjers kode for at opfylde tutorgruppens emailbehov.

Django er et velunderstøttet web framework, og der er mange features, der kan
implementeres med få linjers kode. Vil du lave et RSS feed? Okay - du skal lave
en 40 linjers subclass af \texttt{Feed} og pege på den i din \texttt{urls.py}.
Den største udfordring er ofte at finde ud af præcis hvad man skal skrive.

\chapter{Webgruppens historie}

% print '\n'.join([str(t.year)+' '+t.profile.user.get_full_name() for t in  Tutor.objects.filter(groups__handle='web')])

\section*{2013: Mathias Rav}

\begin{itemize}
  \item To be determined
\end{itemize}

\section*{2012: Mathias Rav}

\begin{itemize}
  \item Emil Bremer Orloff
  \item Frederik Jerløv
  \item Jacob Damgaard Jensen
  \item Morten Henriksen Birk
  \item Thomas Faurbye Nielsen
  \item Tobias Ansbak Louv
\end{itemize}

\section*{2011: Peter Urbak}

\begin{itemize}
  \item Donni Noergaard
  \item Peter Urbak
  \item Randi Katrine Hilleroe
  \item Steffen Videbæk Petersen
  \item Thomas Faurbye Nielsen
  \item Tobias Ansbak Louv
\end{itemize}

\section*{2010: Frederik Mogensen}

\begin{itemize}
  \item Peter Urbak
  \item Steffen Videbæk Petersen
\end{itemize}

\section*{2009: Steffen Videbæk Petersen}

\begin{itemize}
  \item Adam Tulinius
  \item Christian Kraglund Andersen
  \item Frederik Mogensen
  \item Henrik Knakkegaard Christensen
  \item Jakob Grauenkjær Thomsen
  \item Mikkel Vester
  \item Morten N. Pløger
\end{itemize}

\section*{2008: Jakob Grauenkjær Thomsen}

\begin{itemize}
  \item Anders Ingemann
  \item Casper Bach Poulsen
  \item Mads Baggesen
  \item Mads Ravn
  \item Mikkel Vester
  \item Peter Kristensen
  \item Steffen Videbæk Petersen
\end{itemize}

\end{document}
